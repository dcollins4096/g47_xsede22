\section{Simulation Plan}
\label{sec.plan}

Here we will outline the simulations to be performed for each of the projects.

The total cost for one simulation is determined by multiplying the cost for a
single zone-update by the number of zones and the number of updates.  Thus, $SU
= \suzu Z U$, where $SU$ is the total cost, $\suzu$ is cost in
$SU$-per-zone-update, $Z$ is the number of zones, and $U$ is the number of
updates.
$\suzu$ is determined by the total
speed of all of the physics packages employed for each time step, and is
different for each simulation suite. The choice of physics package was motivated
in Section \ref{sec.background}, and the measurement of $\suzu$ is presented in
the Scaling and Performance document.  The estimate of the number of zones, $Z$,
comes from a combination of the simulation domain and the expected number of
zones for each level of the simulation.  The number of updates, $U$, is found as
$U=T/\Delta t$, where the total simulation time is $T$ and the size of the
timestep is $\Delta t$.  $T$ is determined by the physics
problem.  For all of our simulations, the size of the time step $\Delta t$ is
determined by a standard Courant condition, that is a wave cannot cross half of
one zone in a timestep.  That is, 
\begin{align}
\Delta t = \eta \frac{\Delta
x}{v_{\rm{signal}}} \label{eqn.cfl}
\end{align}
, and $\eta < 0.5$.  We determine $v_{\rm{signal}}$, the
fastest signal speed, from preliminary studies, use Equation \ref{eqn.cfl} to
determine the number of steps on each level.
For each suite of simulations, we determine the performance $\suzu$ from
preliminary studies that closely mirror the simulation structure and physics
packages.  

\subsection{Simulations: Turbulent Energy}
\label{subsec.turb_accounting}

Lots of big PPM runs.


\subsection{Simulations: Star Formation}
\label{sec.sims_cores}

Big tracer runs.


\subsection{Simulations: Galaxies}
\label{sec.sims_galaxies}

Really, this is "get something off the ground."


\subsection{Simulations: Foregrounds}
\label{sec.sims_cmb}

Big magnetic simulations.


                                                                                                                                                                       
                                                                                                                                                                       
                                                                                                                                                                       
                                                                                                                                                                       
                                                                                                                                                                       
\begin{table} \begin{center} \caption{Allocation request.  Cost for each simulation, $SU$, is computed from
the cost per zone-update, $\suzu$, the number of zones $N_Z$, and the number of
timesteps $N_U$.
For the \nameTurbulence\ and \nameCMB\ suites, the
simulations are itemized by Mach number and \alf\ Mach number, $M_s$ and $M_a$,
which affect the total time, T, and timestep size $\Delta T$.  
The cost of the AMR simulations, \nameCores\ and \nameGalaxies\, is estimated
for each level, $\ell$, by estimating the volume fraction, $f_\ell$, covered on level
    $\ell$, time step $\Delta t$.  Long term disk usage is estimated as $N_Z$
    times the number of fields for each simulation.  More details are given in
    the text }
 \label{table2}                                                                                                                                                               
\begin{tabular}{l               c               r               r               r                       r                       r               r               r       }       
   suite       &   $M_s$       &$f_\ell$       &     \Nz       &       T               &$\Delta T$               &     \Nu       &   \suzu       &      SU             \\
  \hline                                                                                                                                                               
\nameTurbulence       &     0.5       &       1       &1.1\sci{9}       &     5.0               &7\sci{-6}               &7.0\sci{5}       &2.0\sci{-11}       &1.5\sci{4}             \\
\nameTurbulence       &     1.0       &       1       &1.1\sci{9}       &     2.5               &5\sci{-6}               &4.7\sci{5}       &2.0\sci{-11}       &1.0\sci{4}             \\
\nameTurbulence       &     2.0       &       1       &1.1\sci{9}       &     1.3               &4\sci{-6}               &3.5\sci{5}       &2.0\sci{-11}       &7.6\sci{3}             \\
\nameTurbulence       &     4.0       &       1       &1.1\sci{9}       &     0.6               &2\sci{-6}               &2.9\sci{5}       &2.0\sci{-11}       &6.3\sci{3}             \\
\nameTurbulence       &     7.0       &       1       &1.1\sci{9}       &     0.4               &1\sci{-6}               &2.7\sci{5}       &2.0\sci{-11}       &5.8\sci{3}             \\
  \hline                                                                                                                                                               
               &               &               &               &                       &                       &               &      SU       &4.5\sci{4}             \\
               &               &               &               &                       &                       &               &    Disk       &5.6\sci{3}             \\
   suite       &  $\ell$       &$f_\ell$       &     \Nz       &       T               &$\Delta T$               &     \Nu       &   \suzu       &      SU             \\
  \hline                                                                                                                                                               
\nameCores       &       0       &1.0\sci{0}       &1.3\sci{8}       &       1     Myr       &5\sci{-3}     Myr       &2.2\sci{2}       &6.3\sci{-11}       &1.8\sci{0}             \\
\nameCores       &       1       &4.6\sci{-1}       &4.9\sci{8}       &       1     Myr       &2\sci{-3}     Myr       &4.4\sci{2}       &6.3\sci{-11}       &1.4\sci{1}             \\
\nameCores       &       2       &8.3\sci{-2}       &7.1\sci{8}       &       1     Myr       &1\sci{-3}     Myr       &8.7\sci{2}       &6.3\sci{-11}       &3.9\sci{1}             \\
\nameCores       &       3       &1.3\sci{-2}       &8.7\sci{8}       &       1     Myr       &6\sci{-4}     Myr       &1.7\sci{3}       &6.3\sci{-11}       &9.5\sci{1}             \\
\nameCores       &       4       &1.8\sci{-3}       &1.0\sci{9}       &       1     Myr       &3\sci{-4}     Myr       &3.5\sci{3}       &6.3\sci{-11}       &2.2\sci{2}             \\
  \hline                                                                                                                                                               
               &               &               &               &                       &                       &               & per sim       &3.7\sci{2}             \\
               &               &               &               &                       &                       &               &      SU       &1.1\sci{3}             \\
               &               &               &               &                       &                       &               &    Disk       &2.0\sci{4}             \\
   suite       &$M_{s,a}$       &$f_\ell$       &     \Nz       &       T               &$\Delta T$               &     \Nu       &   \suzu       &      SU             \\
  \hline                                                                                                                                                               
\nameCMB       &     1,1       &       1       &1.1\sci{9}       &       3               &4\sci{-5}               &6.9\sci{4}       &6.2\sci{-11}       &4.6\sci{3}             \\
\nameCMB       &     1,5       &       1       &1.1\sci{9}       &       3               &1\sci{-5}               &2.1\sci{5}       &6.2\sci{-11}       &1.4\sci{4}             \\
\nameCMB       &     5,1       &       1       &1.1\sci{9}       &     0.6               &1\sci{-5}               &4.2\sci{4}       &6.2\sci{-11}       &2.8\sci{3}             \\
\nameCMB       &     5,5       &       1       &1.1\sci{9}       &     0.6               &9\sci{-6}               &6.9\sci{4}       &6.2\sci{-11}       &4.6\sci{3}             \\
  \hline                                                                                                                                                               
               &               &               &               &                       &                       &               &      SU       &2.6\sci{4}             \\
               &               &               &               &                       &                       &               &    Disk       &1.7\sci{4}             \\
   suite       &  $\ell$       &$f_\ell$       &     \Nz       &       T               &$\Delta T$               &     \Nu       &   \suzu       &      SU             \\
  \hline                                                                                                                                                               
\nameGalaxies       &       0       &1.0\sci{0}       &1.7\sci{7}       &       1     Gyr       &4\sci{-4}     Gyr       &2.8\sci{3}       &3.0\sci{-10}       &1.4\sci{1}             \\
\nameGalaxies       &       1       &1.0\sci{0}       &1.3\sci{8}       &       1     Gyr       &2\sci{-4}     Gyr       &5.7\sci{3}       &3.0\sci{-10}       &2.3\sci{2}             \\
\nameGalaxies       &       2       &1.2\sci{-1}       &1.3\sci{8}       &       1     Gyr       &9\sci{-5}     Gyr       &1.1\sci{4}       &3.0\sci{-10}       &4.5\sci{2}             \\
\nameGalaxies       &       3       &2.9\sci{-2}       &2.5\sci{8}       &       1     Gyr       &4\sci{-5}     Gyr       &2.3\sci{4}       &3.0\sci{-10}       &1.7\sci{3}             \\
\nameGalaxies       &       4       &4.2\sci{-3}       &2.9\sci{8}       &       1     Gyr       &2\sci{-5}     Gyr       &4.5\sci{4}       &3.0\sci{-10}       &3.9\sci{3}             \\
\nameGalaxies       &       6       &3.9\sci{-4}       &1.7\sci{8}       &       1     Gyr       &6\sci{-6}     Gyr       &9.1\sci{4}       &3.0\sci{-10}       &9.4\sci{3}             \\
\nameGalaxies       &       9       &3.9\sci{-5}       &8.8\sci{7}       &       1     Gyr       &7\sci{-7}     Gyr       &1.8\sci{5}       &3.0\sci{-10}       &3.9\sci{4}             \\
  \hline                                                                                                                                                               
               &               &               &               &                       &                       &               & per sim       &5.4\sci{4}             \\
               &               &               &               &                       &                       &               &      SU       &2.2\sci{5}             \\
               &               &               &               &                       &                       &               &    Disk       &2.2\sci{4}             \\
  \hline                                                                                                                                                               
  \hline                                                                                                                                                               
               &               &               &               &                       &                       &               &      SU       &2.9\sci{5}             \\
               &               &               &               &                       &                       &               &    Disk       &6.4\sci{4}               
\end{tabular}                                                                                                                                                               
\end{center}                                                                                                                                                               
\end{table}                                                                                                                                                                
