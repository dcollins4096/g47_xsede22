\section{Simulation Plan}
\label{sec.plan}

Here we will outline the simulations to be performed for each of the projects.

The total cost for one simulation is determined by multiplying the cost for a
single zone-update by the number of zones and the number of updates:
\begin{align}
SU = \suzu N_Z N_U,
\end{align}
where $SU$ is the total cost, $\suzu$ is cost in
$SU$-per-zone-update, $N_Z$ is the number of zones, and $N_U$ is the number of
updates.
$\suzu$ is determined by the total time for one time step using 64 cores per
node.  This changes for each suite as the physics packages involved are
different, and the overhead from the AMR hierarchy is different.  Overhead is
accounted for in $\suzu$ by performing simulations that use the target physics
packages and appropriate AMR structures.  More details can be seen in the
Scaling document.

The choice of physics packages for each suite is motivated
in Section \ref{sec.background}, and the measurement of $\suzu$ is presented in
the Scaling and Performance document.  The estimate of the number of zones, $N_Z$,
is determined by the target resolution for the simulation.  For the fixed
resolution simulations, this is simply the number of zones.  For the AMR
simulations, the actual number of zones created is a product of the chaotic
processes in the system.  For these we use approximate the covering fraction
with previous simulations.
The number of updates, $U$, is found as
$U=T/\Delta T$, where the total simulation time is $T$ and the size of the
timestep is $\Delta T$.  $T$ is determined by the physics
problem.  The size of the time step $\Delta T$ is
determined by a standard Courant condition, that is a wave cannot cross half of
one zone in a timestep, 
\begin{align}
\Delta T = \eta \frac{\Delta
x}{\vsignal}, \label{eqn.cfl}
\end{align}
and $\eta < 0.5$.  We determine $v_{\rm{signal}}=c_s+v_{\rm{max}}$
 as the sum of the sound speed and the max velocity, from preliminary studies, and then use use Equation \ref{eqn.cfl} to
determine the number of steps on each level.

Both the \nameTurbulence\ and \nameCMB\ simulations are fixed resolution and
employ only the random forcing physics package.  The former will use the hydro solver
PPM, and the later will use our MHD solver, which has a slightly higher cost.  Both will be
run at $1024^3$.  The total time, $T$, is \red{5} shock-crossing times, so $T =
5 L/\mach$, where $L$ is the size of the driving pattern.  The
timestep, $\Delta T$, also decreases with Mach number as Equation \ref{eqn.cfl},
and is determined measuring the signal speed \vsignal from
previous fully developed turbulence simulations and rescaling with the Mach
number.

The timing for the \nameCores\ and \nameGalaxies\ suite is done in an identical manner, but the
number of zones $N_Z$ is now dynamically determined by the portion of the flow
that is turning into stars.  This is a chaotic process, so formally impossible
to predict
However, it can be expected to be roughly similar to
previous simulations, so we estimate the covering fraction from those.   The timestep per level is typically half that of the level
above, so we estimate the signal speed from previous simulations and estimate
the timestep for each level with Equation \ref{eqn.cfl}.  Table \ref{table2}
shows the estimated covering fraction, timestep, and ultimate SU cost for one
simulation.  

The \nameCores\ simulations will have $512^3$ root grid zones and $1024^3$
particles, as well as 4 levels of AMR.  The refinement will be based on the
density.  It will use the isothermal MHD solver, the gravity solver, and the
particle update machinery.  The net cost per zone update for this combination of
physics solvers and a similar AMR structure to the production simulations is
discussed in the Scaling document, and seen in the $\suzu$ column of Table
\ref{table2}.  We will perform three of these simulations.

The \nameGalaxies\ suite will restrict the AMR to the disk of the galaxy, and
use a tower of refinement of 1/2 the length of the box to separate the outer
part of the CGM at 1.3 Mpc from the small star forming regions in the disk.  The
first 5 levels will be static nested AMR levels, the final 4 will be dynamic.
For each level we approximate that about 10\% coverage of the parent level.
This is by construction for the first 5, and from experience with similar
simulations for the final 4.  These simulations will use the MHD solver, the
gravity solver, the particle machinery for the star particles, heating and
cooling of the gas, and supernovae feedback.  We have performed a preliminary
simulation using 5 levels to determine the cost per timestep, \suzu, and
anticipated signal speed, \vsignal, to determine the timestep size. The results
can be seen in Table \ref{table2}. 

Table \ref{table2} shows the breakdown of the total request by simulation.  The
\nameTurbulence\ and \nameCMB\ suites are itemized by Mach number, while the
\nameCores\ and \nameGalaxies\ suites are itemized by AMR level.  Shown in that
table is the name; the parameter, either Mach number, level, or Sonic and \alf\
Mach numbers;  the volume fraction; the number of zones $N_Z$; the total
simulation time $T$; the timestep size $\Delta T$; the total number of
zone-updates $N_Z N_U$, the cost per update given the AMR and physics usage; and
finally the total SU cost.

The disk usage is also estimated from the number of zones, $N_Z$.  Each zone
stores a number of fields, $N_F$: 5 for \nameTurbulence (density, 3 velocity, and
energy); 14 for \nameCores\ and
\nameCMB (density, energy, 3 velocity, 3 magnetic fields, 3 electric fields, 3
additional magnetic fields, see \citet{Collins10}); 24 for the \nameGalaxies\
suite (14 for MHD, 10 for additional chemistry fields.)  So the total memory is
8 bytes for all of $N_Z N_F$ fields.  It is listed in Gb in the table.

                                                                                                                                                                       
                                                                                                                                                                       
                                                                                                                                                                       
                                                                                                                                                                       
                                                                                                                                                                       
\begin{table} \begin{center} \caption{Allocation request.  Cost for each simulation, $SU$, is computed from
the cost per zone-update, $\suzu$, the number of zones $N_Z$, and the number of
timesteps $N_U$.
For the \nameTurbulence\ and \nameCMB\ suites, the
simulations are itemized by Mach number and \alf\ Mach number, $M_s$ and $M_a$,
which affect the total time, T, and timestep size $\Delta T$.  
The cost of the AMR simulations, \nameCores\ and \nameGalaxies\, is estimated
for each level, $\ell$, by estimating the volume fraction, $f_\ell$, covered on level
    $\ell$, time step $\Delta t$.  Long term disk usage is estimated as $N_Z$
    times the number of fields for each simulation.  More details are given in
    the text }
 \label{table2}                                                                                                                                                               
\begin{tabular}{l               c               r               r               r                       r                       r               r               r       }       
   suite       &   $M_s$       &$f_\ell$       &     \Nz       &       T               &$\Delta T$               &     \Nu       &   \suzu       &      SU             \\
  \hline                                                                                                                                                               
\nameTurbulence       &     0.5       &       1       &1.1\sci{9}       &     5.0               &7\sci{-6}               &7.0\sci{5}       &2.0\sci{-11}       &1.5\sci{4}             \\
\nameTurbulence       &     1.0       &       1       &1.1\sci{9}       &     2.5               &5\sci{-6}               &4.7\sci{5}       &2.0\sci{-11}       &1.0\sci{4}             \\
\nameTurbulence       &     2.0       &       1       &1.1\sci{9}       &     1.3               &4\sci{-6}               &3.5\sci{5}       &2.0\sci{-11}       &7.6\sci{3}             \\
\nameTurbulence       &     4.0       &       1       &1.1\sci{9}       &     0.6               &2\sci{-6}               &2.9\sci{5}       &2.0\sci{-11}       &6.3\sci{3}             \\
\nameTurbulence       &     7.0       &       1       &1.1\sci{9}       &     0.4               &1\sci{-6}               &2.7\sci{5}       &2.0\sci{-11}       &5.8\sci{3}             \\
  \hline                                                                                                                                                               
               &               &               &               &                       &                       &               &      SU       &4.5\sci{4}             \\
               &               &               &               &                       &                       &               &    Disk       &5.6\sci{3}             \\
   suite       &  $\ell$       &$f_\ell$       &     \Nz       &       T               &$\Delta T$               &     \Nu       &   \suzu       &      SU             \\
  \hline                                                                                                                                                               
\nameCores       &       0       &1.0\sci{0}       &1.3\sci{8}       &       1     Myr       &5\sci{-3}     Myr       &2.2\sci{2}       &6.3\sci{-11}       &1.8\sci{0}             \\
\nameCores       &       1       &4.6\sci{-1}       &4.9\sci{8}       &       1     Myr       &2\sci{-3}     Myr       &4.4\sci{2}       &6.3\sci{-11}       &1.4\sci{1}             \\
\nameCores       &       2       &8.3\sci{-2}       &7.1\sci{8}       &       1     Myr       &1\sci{-3}     Myr       &8.7\sci{2}       &6.3\sci{-11}       &3.9\sci{1}             \\
\nameCores       &       3       &1.3\sci{-2}       &8.7\sci{8}       &       1     Myr       &6\sci{-4}     Myr       &1.7\sci{3}       &6.3\sci{-11}       &9.5\sci{1}             \\
\nameCores       &       4       &1.8\sci{-3}       &1.0\sci{9}       &       1     Myr       &3\sci{-4}     Myr       &3.5\sci{3}       &6.3\sci{-11}       &2.2\sci{2}             \\
  \hline                                                                                                                                                               
               &               &               &               &                       &                       &               & per sim       &3.7\sci{2}             \\
               &               &               &               &                       &                       &               &      SU       &1.1\sci{3}             \\
               &               &               &               &                       &                       &               &    Disk       &2.0\sci{4}             \\
   suite       &$M_{s,a}$       &$f_\ell$       &     \Nz       &       T               &$\Delta T$               &     \Nu       &   \suzu       &      SU             \\
  \hline                                                                                                                                                               
\nameCMB       &     1,1       &       1       &1.1\sci{9}       &       3               &4\sci{-5}               &6.9\sci{4}       &6.2\sci{-11}       &4.6\sci{3}             \\
\nameCMB       &     1,5       &       1       &1.1\sci{9}       &       3               &1\sci{-5}               &2.1\sci{5}       &6.2\sci{-11}       &1.4\sci{4}             \\
\nameCMB       &     5,1       &       1       &1.1\sci{9}       &     0.6               &1\sci{-5}               &4.2\sci{4}       &6.2\sci{-11}       &2.8\sci{3}             \\
\nameCMB       &     5,5       &       1       &1.1\sci{9}       &     0.6               &9\sci{-6}               &6.9\sci{4}       &6.2\sci{-11}       &4.6\sci{3}             \\
  \hline                                                                                                                                                               
               &               &               &               &                       &                       &               &      SU       &2.6\sci{4}             \\
               &               &               &               &                       &                       &               &    Disk       &1.7\sci{4}             \\
   suite       &  $\ell$       &$f_\ell$       &     \Nz       &       T               &$\Delta T$               &     \Nu       &   \suzu       &      SU             \\
  \hline                                                                                                                                                               
\nameGalaxies       &       0       &1.0\sci{0}       &1.7\sci{7}       &       1     Gyr       &4\sci{-4}     Gyr       &2.8\sci{3}       &3.0\sci{-10}       &1.4\sci{1}             \\
\nameGalaxies       &       1       &1.0\sci{0}       &1.3\sci{8}       &       1     Gyr       &2\sci{-4}     Gyr       &5.7\sci{3}       &3.0\sci{-10}       &2.3\sci{2}             \\
\nameGalaxies       &       2       &1.2\sci{-1}       &1.3\sci{8}       &       1     Gyr       &9\sci{-5}     Gyr       &1.1\sci{4}       &3.0\sci{-10}       &4.5\sci{2}             \\
\nameGalaxies       &       3       &2.9\sci{-2}       &2.5\sci{8}       &       1     Gyr       &4\sci{-5}     Gyr       &2.3\sci{4}       &3.0\sci{-10}       &1.7\sci{3}             \\
\nameGalaxies       &       4       &4.2\sci{-3}       &2.9\sci{8}       &       1     Gyr       &2\sci{-5}     Gyr       &4.5\sci{4}       &3.0\sci{-10}       &3.9\sci{3}             \\
\nameGalaxies       &       6       &3.9\sci{-4}       &1.7\sci{8}       &       1     Gyr       &6\sci{-6}     Gyr       &9.1\sci{4}       &3.0\sci{-10}       &9.4\sci{3}             \\
\nameGalaxies       &       9       &3.9\sci{-5}       &8.8\sci{7}       &       1     Gyr       &7\sci{-7}     Gyr       &1.8\sci{5}       &3.0\sci{-10}       &3.9\sci{4}             \\
  \hline                                                                                                                                                               
               &               &               &               &                       &                       &               & per sim       &5.4\sci{4}             \\
               &               &               &               &                       &                       &               &      SU       &2.2\sci{5}             \\
               &               &               &               &                       &                       &               &    Disk       &2.2\sci{4}             \\
  \hline                                                                                                                                                               
  \hline                                                                                                                                                               
               &               &               &               &                       &                       &               &      SU       &2.9\sci{5}             \\
               &               &               &               &                       &                       &               &    Disk       &6.4\sci{4}               
\end{tabular}                                                                                                                                                               
\end{center}                                                                                                                                                               
\end{table}                                                                                                                                                                
