\section{Introduction}
\label{sec.intro}

We are requesting \SUtotal\ SUs on Stampede 2 for the period beginning June 1,
2022.  This allocation will support four projects involving astrophysical
magnetic fields.  The first project (\nameTurbulence) explores analytical
formulae we have developed for isothermal turbulence, which is relevant for many
astrophysical processes, among them the formation of stars.  The second project
(\nameCores) simulations the formation of pre-stellar cores from low density
interstellar clouds.  The third project (\nameCMB) examines the polarized signal
produced by the interstellar medium, which is in the way of our understanding of
the cosmic microwave background (CMB). The fourth projcet (\nameGalaxies)
simulates entire galaxies, in order to understand the growth of the magnetic
field.
This research is supported by two NSF grants.  The first two projects
(\nameTurbulence\ and \nameCores)  
are
supported by NSF AST-1616026, and the third (\nameCMB) is supported by 
NSF AST-2009870.
We are hopful that the \nameGalaxies\ project will be funded by a pending
proposal.


These projects support three graduate students.  Luz Jimenez Vela is working on
the \nameCores\ project; Branislav Rabatin is working on the \nameTurbulence\
and \nameCMB\ projects; and Jacob Strack is working on the \nameGalaxies\
project.


Table \ref{table_1} shows the cost for each project.  Each of the four projects
uses a slightly different physics package, which affects  the cost of the
simulation.  In addition, two of the four projects employ adaptive mesh
refinement (AMR), a technique that adaptively changes the resolution of the
simulation.  This also affects the cost of the simulation.

In Section
\ref{sec.background} we motivate each project. In Section \ref{sec.method} we
describe the compuational tools to be used.  In Section \ref{sec.plan} we
outline the simulations to be run and their projected cost.  



\begin{table}[h]
\begin{center}
    \caption{Summary of simulation projects.  The total node hours and disk
    usage are described in Section \ref{sec.plan}.  The physics packages used in
    eash project and
    adaptive mesh refinement (AMR) structure are described in Section
    \ref{sec.background}}
\begin{tabular}{lrrrr}\label{table1}
Name    &    Node Hours    &    Disk    &    Physics     &    AMR    \\
\hline
Turbulence    &   $1.8\sci{4}$  &   $5.6\sci{3}$  &    Hydro + Driving    &    None    \\
Cores         &   $1.2\sci{4}$  &   $1.6\sci{5}$  &    MHD + Gravity + Particles    &    4 levels, all space    \\
CMB           &   $2.6\sci{4}$  &   $1.7\sci{4}$  &    MHD + Driving    &    None    \\
Galaxies      &   $1.1\sci{4}$  &   $1.1\sci{4}$  &    MHD + Gravity + Cooling  &    8 level nest    \\
\hline
&    $6.6\sci{4}$    &    $1.9\sci{5}$   &        &        \\
\end{tabular}
\end{center}
\end{table}



