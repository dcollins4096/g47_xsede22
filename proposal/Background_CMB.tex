The cosmic microwave background (CMB) is the light leftover from the creation of
the universe.  It has taught us a considerable amount about the structure of the
universe.  To learn more from it, we must understand its polarization.  To see
the polarization of the CMB, we must first understand the polarization of the
interstellar medium (ISM), which is much brighter and in the way.  We will
motivate this project in Section \ref{subsec.cmb_motivate}, and describe the
simulations we will perform in Section \ref{subsec.cmb_sims}

\subsection{Motivation: \nameCMB}
\label{subsec.cmb_motivate}

The CMB is extremely uniform on the sky, a perfect black body with a
temperature of 2.7K. Study of the $\sim \mu$K fluctuations in this temperature, through satellites such as Planck, closed many open
questions about the nature of the universe, such as it's energy content and
eventual fate.  But it still has open questions.

When the universe was very young, it was very small, and also very hot.  So hot
that there were no atoms, only bare protons and electrons (and the occasional
alpha particle.)  Sound waves, triggered by the
beginning event, locally compress and expand the gas, causing very small
temperature fluctuations.  As the universe expands, it cools.  About 400,000
years after the beginning of time, the universe had cooled enough for the electrons and
protons to meet and combine, rather than just scattering off one another.  In a
very short period of time, the universe all became neutral, and photons could
travel great distances instead of bouncing off a nearby electron.  These photons
are the CMB.

The fluctuations in the temperature are small, about a millionth of a Kelvin,
but have allowed us to measure many factors of the universe.  Chief among them
are the curvature, mass density (both dark matter and luminous mass) and dark energy
density.  It has helped prove the acceleration of the expansion of the Universe.
The CMB has helped answer many questions about the Universe.

Yet there are several questions that
have not yet been answered, such as, why is it a single temperature?  The
universe is very large, and the CMB photons have traveled a great distance.  So
great, that the distance light has traveled during the age of the universe, when
viewed at the distance of the CMB, is only the size of the full moon.  So why is
the same temperature everywhere?  One possible answer is an extremely rapid
\emph{inflation} of the universe, where the universe expands from the size of a
proton to the size of the solar system within the first $10^{-16}$.  Such a
violent event would leave a sea of gravitational waves.  These gravitational
waves, being quadrupolar in nature, imprint a polarization on the CMB. To detecting
this polarization is to witness the violent birth of the Universe.

Unfortunately (and also fortunately, but for different reasons) the Galaxy we live in is filled with dust.
This dust, which includes iron and magnesium, lines up perpendicular to the magnetic
field in the galaxy, not unlike iron filings around a bar magnet.  These dust
grains radiate polarized thermal radiation in the microwave and infrared.  This
polarized signal is much brighter than the polarization in the CMB, so must be
removed.  In order to remove it, we must understand the statistical properties
of the ISM.

The Planck satellite \red{ref} measured the polarized sky, and the result can be
seen in Figure \red{PLANCK}.  Statistically, the polarization is described best
by the quantities $E$ and $B$.  The $E$ mode is the amplitude in polarization
that is either parallel to or perpendicular to filamentary structures, while $B$
describes polarization at oblique angles.  It is found that both structures are
distributed over all scales in a power-law fashion, with $E \propto k^{-2.35}$,
where $k$ is wavenumber on the sky. $B$ has a similar exponent but half the
amplitude.  

\red{still kind of ratty.}

Our group has had success in reproducing similar behavior in a few settings,.
In \red{Stalpes} we demonstrated that MHD turbulence can reproduce
similar exponents, with the value of the exponent and amplitude depending on
velocity and magnetic field strength in the turbulence.  In
\citet{Huffenberger20}, we developed a model of the ISM polarization based on
magnetized filaments.   

In the proposed simulations, we will combine these two approaches.  We will
perform a series of driven turbulent boxes, as described in \red{REF}, but this
time with magnetic fields.  We will then use the filament finding tool DISPERSE
\red{disperse} 

The study of the ISM is very rich.  We are embarking of two lines of study.  The
first uses the galaxies of the \nameGalaxies\ project.  We will use the galaxies
of that suite of study to constrain the available polarization signals 


\subsection{Simulations: \nameCMB}
\label{subsec.cmb_sims}

Our previous simulations have indicated that, as far as the foregrounds are
conserned, the ISM is most likely supersonic and \sa.  The primary parameters
that dictate the behavior of the turbulence is the sonice Mach number, \mach,
and the \alf Mach number, \alfmach.  These are the ratio of the r.m.s fluid
speed to the sonic speed and \alf\ speed, respectively.  The \alf\ speed is the
speed a disturbance travels along a magnetic field in a plasma.  As the
charachter of the turbulence changes as both \mach\ and \alfmach\ are increased,
we will perform a minimal 4 step parameter search, with \mach and \alfmach equal
to 1 and 5.  Thus, (\mach, \alfmach) = (1,1), (1,5), (5,1) and (5,5) alternately
varying large and small field and velocity.  These will use the MHD and driving
modules in Enzo.  We will increase the resolution beyond that of our preliminary
runs, to $1024^3$.  We will drive the turbulence for 5 dynamical times, where a
dynamical time is the time for a typical driving pattern to cross the box.  As
the driving pattern is 1/2 the box, $T_{dyn}=0.5/\mach$.  Driving for a number
of dynamical times is important to develop statistically relaxed turbulence, as
well as providing statistics for averaging.  
