The galactic magnetic field can be seen in Figure \ref{fig.planck}, which shows
the 353GHz channel of the Planck satellite.  This figure
shows an all-sky projection of dust in the galaxy, smeared along the direction of
the magnetic field.  Our goal in the \nameGalaxies\ project is to understand the
origin of this magnetic field.   We will We will discuss the background in Section
\ref{subsec.galaxies_motivate}, and describe the simulations we will perform in
Section \ref{subsec.galaxies_sims}

\subsubsection{Motivation: \nameGalaxies}
\label{subsec.galaxies_motivate}

The Milky Way has a large scale magnetic
field of roughly $\sim 5 \mu$G, about 200,000 time weaker than a refrigerator
magnet, but spanning the entire galaxy.   In the previous project, \nameCMB, the
goal is to remove the magnetic field in the sky, while in
the \nameGalaxies\ project the goal is to form the magnetic field.

The origin of this magnetic field is an open question.  There are presently two
known \emph{dynamos}, that is mechanisms to amplify magnetic fields. They differ in
two ways; the length scales over which they act, and the time scales over which
they act.  The fast
dynamo converts turbulent kinetic energy to magnetic energy at small scales, and
produces disordered fields quickly.  The slow dynamo produces large scale fields
slowly, with large scale convective motions. The magnetic field in the Milky Way, as well as other similar galaxies,
shows large scale order, but based on observations of old galaxies, must have been
built up quickly. 

The magnetic field in the Galaxy is largely in one direction, closely following
the spiral arms.  Both
dynamo mechanisms produce a
substantial amount of field in all directions.  Thus to have a field of mostly
one direction, the other directions must be expelled from the galaxy.
(Selectively damping out one component of magnetic field is not possible at
these scales.)
Thus the buoyancy of the gas as
it leaves the face of the disk is important in setting the rate of growth of
the mean field.  Like many problems in physics, the answer depends sensitively
on boundary conditions.

The circum-galactic medium (CGM) is the gas that's outside disk the galaxy, but still
bound to it.  It is extremely hot (millions of Kelvin) and extremely low density
(0.1 $cm^{-3}$) and thus unfortunately difficult to observe.
The purpose of this project is to examine the impact of the circum-galactic
medium (CGM) properties on the dynamo. We expect that an ordered field within
the disk requires a buoyant CGM, so that gas that is expelled from the galaxy by
supernovae continues to rise, rather than falling back down immediately.  

\subsubsection{Simulations: \nameGalaxies}
\label{subsec.galaxies_sims}

The disk of our simulated galaxies will be 500pc thick and 25kpc in radius.  Our
proposed simulation domain will begin at very large scale, $1.3$ Mpc.  This is to
separate the boundary from the region of interest, and to give the CGM a large
enough volume to expand.  This will begin at $256^3$, much smaller than the
other simulations, but this suite of galaxy simulations has much deeper AMR.  We will resolve a nest of refinement grids, each one
1/2 of its parent grid on a side, giving constant number of zones per level.
This will be done for 5 levels.  We will allow the simulation to 
refine for a further 4 levels, based on the local density of the gas.  Nine
levels then gives us 10pc of resolution on the finest
level, so we will resolve molecular clouds by a few zones.  We will have ample
resolution in the disk to study the dynamo action as it occurs, and sufficient
resolution in the CGM to serve as an appropriate boundary.  As we are simulating
the entire galaxy, we can no longer use an idealized isothermal equation of
state as the other simulations do, but will use ISM heating and cooling functions by way of the
tabulated look up using Grackle \citep{Smith17}.  We will perform four such simulations with a variety of models
for the CGM.  One simulation will have a buoyant CGM, one will not, and the
other two are developmental simulations for defining simulation parameters.
Simulations will last for 1Gyr, several orbital timescales for the galaxy.  

These simulations will also be useful in conjunction with the \nameCMB\ project.
The two approaches compliment each other, as the \nameCMB\ simulations will
resolve the turbulence with great detail, but the \nameGalaxies\ simulations
will capture the multiphase nature of the ISM and the large scale morphology.
